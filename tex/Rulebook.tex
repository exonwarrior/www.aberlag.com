\mainlanguage[uk]
\setupbodyfontenvironment[default][em=italic]
\ss % Sans-serif font

\definedescription[description][
	headstyle=normal,location=top,text=\twoheadrightarrow\ ,
	style=normal,align=left,width=\overlaywidth,
	command=\hskip-1.5em,margin=1.5em,
]

\setupinteraction[state=start,focus=standard,color=middlered]
\setupfootnotedefinition[before=\ss]

\setupindenting[medium, always]

\def\gamequote#1{\startalignment[center]\em #1\stopalignment\blank[1em]}

\starttext

% --------------------------------------------------
% Title page
% --------------------------------------------------
\definemakeup[titlepage][align=center,doublesided=no]
\setuplayout[titlepage][backspace=20mm,width=150mm]
\starttitlepagemakeup
	{\switchtobodyfont[3.5em] \bold \italic The AberLAG Rulebook}
	
	\blank[2em]
	
	\bf{Version 0.1} \blank[0em] \tfx{(June 2012)}
\stoptitlepagemakeup

% --------------------------------------------------
% Table of contents and headings
% --------------------------------------------------
\setupcombinedlist[content][alternative=c,interaction=all]
\setuplist[part][style=bold]
\setuplist[chapter][]
\setuplist[section][textstyle=italic]
\setuphead[subsection][number=no]
\completecontent

\setuphead[chapter][align=center,
	before={\hrule height 2pt\blank[0.5em]},
	after={\blank[0.5em]\hrule height 2pt\blank[1em]}]

\setuphead[section][align=right, style=bold]
\setuphead[subsection][align=center, style=boldslanted]

% --------------------------------------------------
\part{Player Rulebook}
% --------------------------------------------------
	
	\chapter{Shootouts}

	The shootout rules apply to all shootouts and scenarios.

		\section{Safety}
		
			The safety rules come above all others - breaking any of them result in a temporary or permanent ban from society activities\footnote{Shootouts, scenarios and other games, but not official meetings.}.

			\description{Moderators} Listen to the moderators, and follow their instructions - especially at the start of scenarios and missions.
			
			\description{Physical contact} Don't get into physical contact - don't push, shove or grab at other players. The exceptions to this are LARP (or similar) melee weapons if they have been explicitly allowed, as described below.
			
			\description{Equipment} All weapons must not look realistic, or be painted in dark colours, ad they must have orange tips. All weapons must not hurt on impact\footnote{Modified or unusual weapons must be checked with the Safety Officer before they are allowed - they must not hurt on impact, or have potential to cause injury.}.
			
			\description{Act sensibly} It may not be covered by the rules. But if could cause danger to another player, or get the game banned: {\em Don't do it.}
		
		\section{General rules}
		
			\description{Start and end times} Games start and end with the blowing of an whistle [{\em sic}], or a call of {\em START} and {\em STOP} from a moderator.
			
			\description{Act with honour, integrity and sportsmanship} Don't cheat, ignore the rules and/or moderators, or do anything else which will (or is likely to) spoil the game for other players.
			
			\description{Inactive or `dead' players} If you are out (i.e. no lives left, inactive, tagged etc), you may not interfere with the people who are still in the game. To signify that you are inactive, hold up either two fingers or your weapon. {\em \quote{Dead men don't talk.}}
			
			\description{Clean up} At the end of games, help collect dropped darts, and once the darts have been collected return them to their owners. Marking your name or a unique symbol on darts you want to make sure you keep is advised.
		
		\section{Tagging and Equipment}
		
		You can tag another player by shooting them with a blaster, touching them with a melee weapon, or hitting them with a thrown weapon (such as a sock-ball). When tagging another player, you should call 'HIT' to clearly inform the player that they have been tagged.

			\description{Blasters} Foam dart/disc launchers (``blasters'') may be used to tag other players. Players can be tagged by shooting them with a direct hit ({\em i.e.} the projectile hitting the player and coming to a full stop). Hits to the feet do not count.
			
			\description{Thrown weapons} Rolled up socks can be used as thrown weapons, and tag players in the same way as blasters.
			
			\description{Melee weapons} LARP-style melee weapons may be used to tag another player, by touching them with the weapon. Melee weapons are restricted - they may only be used in moderated events, and only if the moderators have allowed them. For you to be allowed to use them, you must be able to pull your blows, and must have been checked by the Health and Safety Officer. 
	
	\chapter{Scenarios}
	
		\section{Safety}
		
			\description{Dangerous areas} Tags made in a dangerous area (such as a busy road) are immediately invalidated and should always be avoided. Vehicles also count as a dangerous area, and no tags can be made to or from a vehicle (i.e. You can not shoot from a vehicle, or shoot at a player in a vehicle). Players should always avoid these areas.
			
			\description{Safe areas} Some areas are ``safe areas'' where the game is permanently suspended. Blasters should be concealed, and no players may be stunned or tagged. Safe areas include all indoor areas (especially public areas such as cafeterias and shops), workplaces and places of worship.
		
		\section{Markers}
		
			Scenarios usually mark teams with a marker - generally an armband or bandanna of a certain colour. The marker should be visibly worn at all times when you are not in a safe zone. You may only tag players wearing a marker - assume people without a marker are not playing\footnote{If you think a player is cheating by removing their marker, contact the moderators - do not act as if they are in the game.}.

% --------------------------------------------------
\part{Game rules}
% --------------------------------------------------

	\chapter{Shootouts}

		Rules for some of the common shootouts.
		
		\section{Prison}
			
			\gamequote{Teams attempt to capture all members of the opposing team.}
			
			Each team is given a zone - their `prison'. This is where they spawn at the start of the game.
			
			When a player is tagged, they are sent to the prison of the opposing team\footnote{In games where there are more than two teams, they go to the prison of the player that killed them}. While in prison, players can do nothing but sit/stand - however, they are released when touched by another players hand.
			
			A team is out of the game once all of their members are in prison. The last team remaining wins the game.
			
		\section{Destruction}
			
			\gamequote{Based on games such as Counter-Strike, Destruction involves teams attempting to either detonate a bomb in a location or disarm the bomb.}
	
	\chapter{Tournaments}
	
	

	\chapter{Scenarios}

		\section{Humans vs Zombies (HvZ)}
		
		Swarm zombies - act as a re-spawn point for zombies. Limit to the alpha zombie? Needs a very clear marker - bright <zombie team color> sash?
		
		Alpha zombie - releases a cure when killed?

% --------------------------------------------------
\part{Moderator Rulebook}
% --------------------------------------------------

	\chapter{Moderating Shootouts}
	
	\chapter{Moderating Scenarios}

\stoptext