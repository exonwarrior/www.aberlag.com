\documentclass{article}
\usepackage[cm]{fullpage}
\renewcommand{\familydefault}{\sfdefault}

% Hyperlinks
\usepackage[usenames,dvipsnames]{xcolor}
\usepackage[pdftitle={The Aberystwyth Live Action Games Society Constitution}, pdfauthor={The AberLAG Committee}, pdfcreator={Sam Clements}]{hyperref}
\hypersetup{colorlinks=true, linktocpage=true, linktoc=all, linkcolor=Black, urlcolor=Black} 

% Commands
\newcommand{\society}{Aberystwyth Live Action Games Society}
\newcommand{\sref}[1]{\hyperref[#1]{Section \ref*{#1}}}
\newcommand{\aref}[1]{\hyperref[#1]{Appendix \ref*{#1}}}
\newcommand{\itemtitle}[1]{\textbf{#1} \hfill \\}

\begin{document}

\title{The Aberystwyth Live Action Games Society Moderator Rulebook}
\author{AberLAG committee}
\date{May 2012}
\maketitle{}

\tableofcontents
\newpage

\section{Shootouts}
		
	\subsection{Moderation}
	
		Moderators are appointed by the committee, and are able to oversee shootouts. Shootouts should have 1 active moderator (i.e. moderating the game, and not taking part) for each full ten players at the activity.
		
		Moderators can temporarily change rules for the duration of the shootout, though such rules must be decided upon by the committee before becoming permanent.

		Moderators have several shouts or `calls' they can use while running a game.
	
		\begin{itemize}
			\item \textbf{PAUSE} / \textbf{PLAY} ({\em or a single whistle}) - Used to temporarily pause a game. Players should lower weapons, and move out of the way if they need to. The game will continue once the issue has been resolved, and players have moved back to the positions they were in before the game was paused.
			
			\item \textbf{START} / \textbf{STOP} ({\em or two whistles}) - Used to start and end a game.
			
			\item \textbf{HALT} ({\em or three whistles}) - Used to stop play completley, in case of an emergency such as injury - similar to the call of MAN DOWN found in some LARPs. Unlike the other calls, this can be use by any player and not just active moderators.
		\end{itemize}
		
		\noindent The moderators should also:
			
		\begin{itemize}
			\item Organise the matches played, and the teams for them.
			\item Ensure that all darts are cleared up at the end of matches, and borrowed equipment returned to it's owner.
			\item Ensure that no unsafe equipment is being used.
			\item Arbitrate arguments between players.
		\end{itemize}

\section{Scenarios and tournaments}

	Scenarios and tournaments will have a set group of moderators, which must include at minimum a Head and Vice-Head moderator (tournaments only require a Head moderator, not both). They have the same powers as 'normal' moderators, though the positions are specific to the event. These moderators must beable to help out during missions and other periods of heavy activity during the event, and take a large role in ensuring it runs smoothly.
	
	\subsection{Head Moderator}
		The Head Moderator and Vice Head Moderator take responsibility for the event for it's duration, and have the final say on disputes over rules (for the duration of the event). In the case of both the Head and Vice Head Moderators not being able to attend all or part of an event, they should decide upon another person to take the role during their absence. They are also completley restricted from taking part in the game, unlike the other moderators.

	\subsection{Indoor and outdoor games}
		Scenarios should generally run ``outside'' (in the context of rules for off-limits areas). Any scenarios that wish to run ``indoors'' must have the full permission of the committee.
	
	\subsection{Tournaments}
		Moderators should be on hand for each match in a tournament, to arbitrate any debate and to ensure the match is fair. They should take care to be unbaised, and should let another moderator take their role if they feel they cannot do so.

\end{document}