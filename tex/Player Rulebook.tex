\documentclass{article}
\usepackage[cm]{fullpage}
\renewcommand{\familydefault}{\sfdefault}

% Hyperlinks
\usepackage[usenames,dvipsnames]{xcolor}
\usepackage[pdftitle={The Aberystwyth Live Action Games Society Constitution}, pdfauthor={The AberLAG Committee}, pdfcreator={Sam Clements}]{hyperref}
\hypersetup{colorlinks=true, linktocpage=true, linktoc=all, linkcolor=Black, urlcolor=Black} 

% Commands
\newcommand{\society}{Aberystwyth Live Action Games Society}
\newcommand{\sref}[1]{\hyperref[#1]{Section \ref*{#1}}}
\newcommand{\aref}[1]{\hyperref[#1]{Appendix \ref*{#1}}}
\newcommand{\itemtitle}[1]{\textbf{#1} \hfill \\}

\begin{document}

\title{The Aberystwyth Live Action Games Society Player Rulebook}
\author{AberLAG committee}
\date{May 2012}
\maketitle{}

\tableofcontents
\newpage

\section{General Rules}
	The general rules apply to all shootouts, scenarios and tournaments. Shootouts require no rules beyond this section, though a few of the game types played are explained in section \ref{shootouts}.

	\subsection{Safety}
		The safety rules come above all others - breaking any of them result in a temporary or permanent ban from society activities.
		
		\begin{itemize}
			\item Listen to the moderators, and follow their instructions - especially at the start of scenarios and missions.
			
			\item Don't get into physical contact - don't push, shove or grab at other players. The exceptions to this are LARP (or similar) melee weapons if they have been explicitly allowed, as described below.
			
			\item All equipment (in particular, blasters and other `weapons') must not look realistic or be painted in dark colours, and must have orange tips. All weapons must not hurt on impact. Modified or unusual weapons must be checked with the Safety Officer before they are allowed - they must not hurt on impact, or have potential to cause injury.
			
			\item Act sensibly - it may not be covered by the rules. But if could cause danger to another player, or get the game banned: {\em Don't do it.}
		\end{itemize}
	
	\subsection{Moderator Calls}
		Moderators use several shouts or `calls' while running a game.
	
		\begin{itemize}
			\item \textbf{PAUSE} / \textbf{PLAY} ({\em or a single whistle}) - Used to temporarily pause a game. Players should lower weapons, and move out of the way if they need to. The game will continue once the issue has been resolved, and players have moved back to the positions they were in before the game was paused.
			
			\item \textbf{START} / \textbf{STOP} ({\em or two whistles}) - Used to start and end a game.
			
			\item \textbf{HALT} ({\em or three whistles}) - Used to stop play completley, in case of an emergency such as injury - similar to the call of MAN DOWN found in some LARPs. Unlike the other calls, this can be use by any player and not just active moderators.
		\end{itemize}
		
	\subsection{Tagging and Equipment}
		
		You can tag another player by shooting them with a blaster, touching them with a melee weapon, or hitting them with a thrown weapon (such as a sock-ball). When tagging another player, you should call 'HIT' ensure the player notices that they have been tagged.

		\begin{itemize}
			\item \textbf{Blasters} - Foam dart/disc launchers (``blasters'') may be used to tag other players. Players can be tagged by shooting them with a direct hit ({\em i.e.} the projectile hitting the player and coming to a full stop). Hits to the feet do not count.
			
			\item \textbf{Thrown weapons} - Rolled up socks (or similar equivalents such as bean bags) can be used as thrown weapons, and tag players in the same way as blasters. As with blasters, they must be soft enough not to hurt on impact.
			
			\item \textbf{Melee weapons} - LARP-style melee weapons may be used to tag another player, by touching them with the weapon. For a player to be allowed to use them, they must be able to pull their blows, and must have been checked by the Health and Safety Officer. Melee weapons ay only be used in scenarios when they have been explictly allowed, but are usually acceptable in shootouts (assuming the player has been checked by the Safety Officer).
		\end{itemize}
	
	\subsection{Sportsmanship}
		While competitive, shootouts and scenarios are a game, and players should ensure they respect the other players taking part.
	
		\begin{itemize}
			\item {\em ``Act with honour, integrity and sportsmanship.''} Don't cheat, ignore the rules or moderators, or do anything else which will (or is likely to) spoil the game for other players.
			
			\item {\em ``Dead men don't talk.''} Inactive players (such those with no lives left, `dead', inactive, tagged, etc.) may not interfere with players who are still active. To signify that you are inactive, hold up either two fingers or your weapon. 
			
			\item At the end of games, help collect dropped darts, and return them to their owners. Marking your name on darts you want to make sure you keep is advised.
		\end{itemize}

\newpage
\section{Scenarios}
	Scenarios refer to larger shootouts, which may take place over multiple days and a larger area. Player should take care to respect their enviroment at all times, and ensure they do not put themselves into danger for the sake of the game. They have some additional rules that do not apply to shootouts.
	
	\subsection{Markers}
		
		Scenarios usually mark teams with a marker - generally an armband or bandanna of a certain colour. The marker should be visibly worn at all times when you are not in a safe zone. You may only tag players wearing a marker - assume people without a marker are not playing\footnote{If you think a player is cheating by removing their marker, contact the moderators - do not act as if they are in the game.}.
	
	\subsection{Inactive players}
		Inactive players may not interfere with the people who are still in the game. This includes communicating information relevant to the game (especially information gained from spectating while inactive) to the active players. However, the exception to this is that you may communicate a) that you have died, and b) the location of your death.
	
	\subsection{Camping}

		Camping is defined as ``{\em blocking an entrance to a safe area, such that a player cannot enter or exit the building without an immediate risk of being tagged''}. Players who are camping may be shot at from inside the safe area. Tags made by campers having the possibility of being invalidated or otherwise punished.
	
	\subsection{Safe areas}
	
		Some areas are ``safe areas'' where the game is permanently suspended. Blasters should be concealed, and no players may be stunned or tagged. This is to avoid disruption to members of the public, and danger to anyone involved.
		
		Tags made in a dangerous area (such as a busy road) are immediately invalidated and should always be avoided. Players should always avoid these areas.
		
		\subsubsection{Safe areas for outside games}

			All indoor areas (especially public areas such as cafeterias and shops) are off-limits. Vehicles and dangerous areas also count as safe, as well as any location that is off limits for an indoor game.
		
		\subsubsection{Safe areas for indoor games}

			Some games may allow indoor combat, as well as outdoor combat (generally games without open warfare). For these games, off-limits areas include:

			\begin{itemize}
				\item Lecture theatres when a lecture is in progress or within 10 minutes of a lecture.
				\item Restaurants, canteens and cafes, as well as any shops.
				\item Libraries and study areas.
				\item Bathrooms and changing rooms.
				\item Personal rooms or offices are off limits - unless you have been invited in by the owner\footnote{Often known as the Vampire rule}.
				\item Places of worship and workplaces.
				\item Roads, vehicles, or other areas likely to cause danger to players.
			\end{itemize}

\newpage
\section{Tournaments}
	This section explains the tournament rules.

% \appendix
\section{Shootouts}
\label{shootouts}
	This section explains some of the common games played in shootouts.
	
	
	\subsection{Prison}
		Each team is given a zone - their `prison'. This is where they spawn at the start of the game. During the game, teams attempt to capture all members of the opposing team(s). When a player is tagged, they are 'captured' and sent to the prison of the player that killed them. While in prison, players can do nothing but sit/stand. Captured players are released when touched by another player of any team. A team is out of the game once all of their members are in prison. The last team remaining wins the game.
	
\end{document}